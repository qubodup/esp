\documentclass[a4paper, 11pt]{article}
\usepackage[utf8]{inputenc}
\usepackage[english]{babel}
\usepackage[T1]{fontenc}
\usepackage{datetime}
\usepackage{textcomp}
\usepackage{listings}
\usepackage[usenames,dvipsnames]{color}
\usepackage[margin=1in]{geometry}
\usepackage{fancyhdr}
\usepackage{lastpage}
\usepackage[colorlinks=true,urlcolor=blue,linkcolor=blue]{hyperref} 
\usepackage{upquote}
\usepackage[scaled=1]{beramono}
\setlength{\parindent}{0pt}



\def\shortyear#1{\expandafter\shortyearhelper#1}
\def\shortyearhelper#1#2#3#4{#3#4}
\newdateformat{specialdate}{\shortyear{\the\year}.\twodigit{\THEMONTH}.\twodigit{\THEDAY}}
\renewcommand{\headrulewidth}{0pt}
\pagestyle{fancy}
\lhead{}

\date{\vspace{-5ex}}
\chead{}
\rhead{}
\lfoot{Introduction to LÖVE - \date{\specialdate\today}}
\cfoot{\thepage/\pageref*{LastPage}}
\rfoot{\href{http://qubodup.itch.io/startgamedev}{qubodup.itch.io/startgamedev}}
\definecolor{darkgray}{rgb}{0.45, 0.45, 0.45}
\definecolor{lightgray}{rgb}{0.9, 0.9, 0.9}

\lstdefinestyle{colorstyle}
{
  numbers          = left,
  language         = {[5.2]Lua},
  basicstyle       = \ttfamily,
  showstringspaces = false,
  commentstyle     = \itshape\color{darkgray},
  numberstyle      = \tiny,
  identifierstyle  = \color{blue},
  keywordstyle     = \color{magenta},
  stringstyle      = \color{red},
  xleftmargin      = 0em,
  rulecolor        = \color{lightgray},
  frame            = l,
  framexleftmargin = .2em
}
\title{\vspace{-8ex}Start Gamedev Introduction to LÖVE\vspace{-1ex}}

\author{\copyright{}2015-2016 Fabian Gerhard, Iwan Gabovitch (\href{http://qubodup.itch.io/startgamedev}{qubodup.itch.io/startgamedev})\\
Licensed under a \href{http://creativecommons.org/licenses/by-sa/4.0/}{Attribution-ShareAlike 4.0 International License}}

\lstset{style=colorstyle}

\exhyphenpenalty=10000\hyphenpenalty=10000
\widowpenalty=10000
\raggedbottom
\clubpenalty=10000
\sloppy

\begin{document}

\maketitle
\thispagestyle{fancy} % footer on 1st page

\section{Prepare}

\begin{enumerate}
  \item Extract \textbf{StartGamedev} and open the text editor using the \texttt{open-editor} file.
  \item Read the tasks, type the code (\textit{source code}) and test the results.
  \item Inside one task (e.g. 1.1), continuously expand the code. When you begin a new task (e.g. 1.2), clear the code in your text editor first.
  \item Functions (e.g. \texttt{function love.draw() ... end}) should only appear once.
  \item Use the Tab key to indent (left of Q). Keep your code readable.
  \item Functions, loops and conditions end with \texttt{\texttt{end}}. The lines above \texttt{\texttt{end}} are the body.
  \item Your line numbers can differ from the numbers in the tasks.
\end{enumerate}

\section{Better than paper: draw in LÖVE}

\subsection{Your favourite rectangle}

A rectangle at position x=100, y=200. 300 pixels in width and 150 in height.
\lstinputlisting{code/intro-rectangle.lua}

\begin{enumerate} 
\item Move the rectangle.
\item The screen is of size 800,600. Align the rectangle with the upper right corner. 
\item Replace \texttt{''fill''} with \texttt{''line''}, what happens?
\item Draw a second rectangle somewhere else. Copy only line 2.
\item Make the screen white.
\end{enumerate}

\subsection{Two rectangles}

\lstinputlisting{code/intro-2rectangles.lua}

\begin{enumerate} 
\item Change numbers in line 2. What happens?
\item This representation of colors using three numbers (0-255) is called RGB (Red-Green-Blue). Make the smaller rectangle blue. 
\item Move the rectangles so that they overlap. Which rectangle is in front?
\item Swap lines 3 and 5. What changed?
\end{enumerate}

\subsection{Some lines}

\lstinputlisting{code/intro-lines.lua}

\begin{enumerate} 
\item Move the rectangle. Adjust the lines accordingly.
\item This was tedious. Variable can do this automatically for us! Read on.
\end{enumerate}

\subsection{Variables}

\lstinputlisting{code/intro-variables.lua}

\begin{enumerate} 
\item What would happen if you changed \texttt{x} and \texttt{y}?
\item Change line 2 to: \texttt{y = x}. What does this mean?
\item Change line 2 back to: \texttt{y = 200}. Change line 1 to: \texttt{x = y}. You will get an error. Can you correct the code?
\item Introduce a variable for the width of the rectangle.
\end{enumerate}

\section{Interaction}

\subsection{A moving picture}

\lstinputlisting{code/intro-interaction.lua}

\begin{enumerate} 
\item Try clicking the game. Something should happen.
\item Make the box go backwards.
\item Make the box go upwards.
\item Make the box bigger on mousepress.
\end{enumerate}

\subsection{Asking for the right click/touch}

\lstinputlisting{code/intro-control.lua}

\begin{enumerate} 
\item Where do you have to click to move the rectangle?
\item Copy line 11 but change it to reduce \texttt{a} when \texttt{dir == "left"}.
\item Let the box touch the border but let it go no further (e.g. add \texttt{and a < 500} before \texttt{then}).
\end{enumerate}

\subsection{It needs to do things on its own}

\lstinputlisting{code/intro-update.lua}

Everything inside the \texttt{love.update} block is executed 60 times per second.
\newline

\begin{enumerate} 
\item Stop the box at the top. \texttt{if y > 200 then} ... \texttt{end} or similar might help.
\item At the top of the code insert \texttt{velocity = 1}. Let the box move with \texttt{y = y - velocity} instead of \texttt{y = y - 1}.
\item Reduce \texttt{veolcity} continuously by 0.01. This simulates gravity.
\item Increase \texttt{velocity} when you click it (use \texttt{love.mousepressed()}).
\item Stop the box at the bottom of the screen.
\item Print the velocity to screen with \texttt{love.graphics.print(velocity,10,10)}
\item Set \texttt{velocity} to 0 when the rectangle touches the top border.
\item Give the player a goal. Notify the player when they reached that goal. Example: A cheap parking game. Draw a line at height 100. Change the color of the box, \texttt{if 0.5 > velocity and velocity > -0.5 and 105 > y and velocity > 95 then}.
\end{enumerate}

\subsection{Loops}

The while-loop executes the program written into it (its body) as long as its condition \texttt{y < 500} is true.

\lstinputlisting{code/intro-loop.lua}

\begin{enumerate}
\item Switch lines 7 and 8. Do you understand what difference this makes?
\item Move line 4 in line 2. The screen should go black. Why is that?
\item Revert your changes, then draw more, but smaller rectangles vertically using the while loop.
\item Introduce a new variable \texttt{z = 0}. Add a new while loop. Make it increase \texttt{z} and let it draw some rectangles horizontally.
\item Move \texttt{z = 0} and the new while loop inside the old while loop. Now every time \texttt{y} is increased your program goes through all your z values! Use this to make a checkerboard. 
\item Let the whole checkerboard be moved by clicking the mouse.
\end{enumerate}

\subsection{Lists}
\lstinputlisting{code/intro-array.lua}
\texttt{a} ist a list (table). \texttt{a[1]} equals \texttt{100}. Here, \texttt{1} is the \textit{index} of \texttt{100} in \texttt{a}.
\newline

\begin{enumerate}
\item Add a number to the list \texttt{a}. \texttt{\#a} is the length of a. Draw all 4 elements.
\item \texttt{a[5] = 5 * 10} sets a new element. Use a while loop to let \texttt{a} be \texttt{\{10,20,....,200\}}.
\item \texttt{a[\#a+1] = v} is the same as adding \texttt{v} to the list. Whenever the mouse is clicked, add the x-coordinate of the click to the list.
\end{enumerate}

\iffalse
\subsection{Functions}
Functions calculate an output to a given input. They can also output nothing and just execute some code based on the input. They can also ignore their input and just be a fancy name for a block of code.
love.draw, love.update and love.mousepressed are special functions which are defined by you but executed by the game engine.\footnote{Executing a function is usually called \textit{calling a function}}

\lstinputlisting{code/intro-functions.lua}

\begin{enumerate}
\item twosquares(50,150) executes the code from lines 2-4 with x being 50 and y being 150. The values \textit{returned} in line 4 are saved in x1,y1. \newline
      Draw another pair of squares at the position returned by twosquares(x1,y1).
\item Create a new function called foursquares(x,y). This function should draw 4 squares all having some distance between them. You can use twosquares.
\item Add this function and call drawatx1y1 in love.draw.\footnote{drawatx1y1() calls the function. drawatx1y1 does not! drawatx1y1 only represents a variable holding the function.}
\begin{lstlisting}
function drawatx1y1()
  love.graphics.rectangle("fill",x1,y1,100,100)
end
\end{lstlisting}
Any code that is executed after line 8 can use x1 and y1, but often you only need a variable inside a block of code.
Sharing that variable with the rest of the code can make it more difficult to understand the code, because small changes can have widespread, hard to track, effects.

Replace line 8 by \textbf{local x1 , y1 = twosquares (50 ,150)}.

Now drawatx1y1 can not access x1 and y1 anymore. The program should crash.

How are variables transfered from love.draw to twosquares?
Can you replicate that strategy to make drawatx1y1 work again?

\item Create a function pointInRectangle(x,y,rectanglex,rectangley,width,height) which returns true when x,y is inside the rectangle drawn by love.graphics.rectangle("fill", rectanglex,rectangley,width,height) and false otherwise.
      Create a moving rectangle with love.update.
      Signify if the player has clicked the moving rectangle using pointInRectangle.
\end{enumerate}

\subsection{Object Orientation}
Objects are some functions and some variables bundled together. They can represent objects of reality like cats and trees.
The functions could then represent what a cat can do while the variables remember attributes of the cat.
This could be concievied as an object. cat[1] could be the age and "lisa"\footnote{"lisa" is a string that could be printed. Strings can be hold in variables just like numbers.} 
\begin{lstlisting}
cat = {3, "lisa", function() love.graphics.print("purr",50,50) end}
\end{lstlisting}
It would be annoying to refer to the name of the cat as cat[2] though.
Lists can actually be more than lists. They can not only associate an index with a value, but any value with any value.
For that we need new syntax.
\begin{lstlisting}
cat = {3, "lisa", function() love.graphics.print("purr",50,50) end}
\end{lstlisting}
is the same as
\begin{lstlisting}
cat = {
  3,
  "lisa",
  function() love.graphics.print("purr",50,50) end
}
\end{lstlisting}
is the same as
\begin{lstlisting}
cat = {
  1 = 3,
  2 = "lisa",
  3 = function() love.graphics.print("purr",50,50) end
}
\end{lstlisting}
but you can do
\begin{lstlisting}
cat = {
  age = 3,
  name = "lisa",
  askforfood = function() love.graphics.print("purr",50,50) end
}
\end{lstlisting}
now cat["age"] is 3 and cat["name"] is lisa.
Beware that cat["age"] is not the same cat[age]. What cat[age] is, depends on what value the variable age has.

One can create a function to make the creation of complicated objects easier.

\begin{lstlisting}
function newCat(age, name)
  return {
    age = age,
    name = name,
    askforfood = function() love.graphics.print("purr",50,50) end
  }
end
\end{lstlisting}
Here's how to give the functions  access the objects variables.\footnote{\textbf{a .. b} appends the string b to a. Instead of cat.askforfood(cat) you can write cat:askforfood(). Its the same.}
\begin{lstlisting}
function newCat(age, name)
  return {
    age = age,
    name = name,
    askforfood = function(thecat)
      love.graphics.print("purr I'm " .. thecat.name ,50,50)
    end
  }
end

cat = newCat(4,"tom")

function love.draw()
  cat.askforfood(cat)
end
\end{lstlisting}

\fi

\lstinputlisting{code/intro-objects.lua}

\end{document}
