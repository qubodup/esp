\documentclass[a4paper, 11pt]{article}
\usepackage[utf8]{inputenc}
\usepackage[german]{babel}
\usepackage[T1]{fontenc}
\usepackage{datetime}
\usepackage{textcomp}
\usepackage{listings}
\usepackage[usenames,dvipsnames]{color}
\usepackage[margin=1in]{geometry}
\usepackage{fancyhdr}
\usepackage{lastpage}
\usepackage[colorlinks=true,urlcolor=blue,linkcolor=blue]{hyperref} 
\usepackage{upquote}
\usepackage[scaled=1]{beramono}
\setlength{\parindent}{0pt}



\def\shortyear#1{\expandafter\shortyearhelper#1}
\def\shortyearhelper#1#2#3#4{#3#4}
\newdateformat{specialdate}{\shortyear{\the\year}.\twodigit{\THEMONTH}.\twodigit{\THEDAY}}
\renewcommand{\headrulewidth}{0pt}
\pagestyle{fancy}
\lhead{}

\date{\vspace{-5ex}}
\chead{}
\rhead{}
\lfoot{LÖVE Einführung - \date{\specialdate\today}}
\cfoot{\thepage/\pageref*{LastPage}}
\rfoot{\href{http://espws.de}{espws.de}}
\definecolor{darkgray}{rgb}{0.45, 0.45, 0.45}
\definecolor{lightgray}{rgb}{0.9, 0.9, 0.9}

\lstdefinestyle{colorstyle}
{
  numbers          = left,
  language         = {[5.2]Lua},
  basicstyle       = \ttfamily,
  showstringspaces = false,
  commentstyle     = \itshape\color{darkgray},
  numberstyle      = \tiny,
  identifierstyle  = \color{blue},
  keywordstyle     = \color{magenta},
  stringstyle      = \color{red},
  xleftmargin      = 0em,
  rulecolor        = \color{lightgray},
  frame            = l,
  framexleftmargin = .2em
}
\lstdefinestyle{bwstyle}
{
  numbers          = left,
  language         = {[5.2]Lua},
  basicstyle       = \ttfamily,
  showstringspaces = false,
  commentstyle     = \itshape\color{darkgray},
  numberstyle      = \tiny,
  identifierstyle  = \bfseries,
  keywordstyle     = \bfseries,
  xleftmargin      = 0em,
  rulecolor        = \color{lightgray},
  frame            = l,
  framexleftmargin = .2em
}

\title{\vspace{-8ex}Einstieg Spiele-Programmierung LÖVE Einführung\vspace{-1ex}}

\author{\copyright{} 2015-2017 Fabian Gerhard, Iwan Gabovitch (\href{http://espws.de}{espws.de})\\
Lizenziert unter einer \href{http://creativecommons.org/licenses/by-sa/4.0/}{Attribution-ShareAlike 4.0 International Lizenz}}

\lstset{style = colorstyle}

\exhyphenpenalty=10000\hyphenpenalty=10000
\widowpenalty=10000
\raggedbottom
\clubpenalty=10000
\sloppy

\begin{document}

\maketitle
\thispagestyle{fancy} % footer on 1st page

\section{Vorbereitung}

\begin{enumerate}
  \item Extrahiere \textbf{StartGamedev} und öffne den Texteditor mithilfe der \texttt{open-editor} Datei.
  \item Lese Aufgaben aufmerksam, tippe Code (\textit{Quelltext}) ab und teste Ergebnisse.
  \item Innerhalb einer Aufgabe (z.B. 1.1) wird der Code kontinuierlich erweitert. Wenn Du eine neue Aufgabe (z.B. 1.2) anfängst, solltest Du deinen Texteditor leeren.
  \item Funktionen (z.B. \texttt{function love.draw() ... end}) dürfen nur ein mal vorkommen.
  \item Benutze die Tabulatortaste um einzurücken (links neben Q), halte Deinen Code lesbar.
  \item Funktionen, Schleifen und Konditionen enden mit \texttt{\texttt{end}}. Die Zeilen davor sind der Körper.
  \item Deine Zeilen können anders durchnummeriert sein, als auf diesem Übungsblatt.
\end{enumerate}

\section{Malen, wie auf einer Leindwand}

\subsection{Dein Lieblingsrechteck}

Ein Rechteck an der Position \texttt{x=100}, \texttt{y=200}. 300 Pixel breit und 150 hoch.
\lstinputlisting{code/intro-rectangle.lua}

\begin{enumerate}
\item Zeichne das Rechteck an einer anderen Stelle.
\item Male das Rechteck ,,genau'' in der rechten oberen Ecke. Das Fenster hat die Größe 800x600.
\item Ersetze \texttt{''fill''} mit \texttt{''line''}. Was passiert nun?
\item Male ein zweites Rechteck woanders. Kopiere nur Zeile 2.
\item Mache das gesamte Fenster Weiß.
\end{enumerate}

\subsection{Zwei Rechtecke}

\lstinputlisting{code/intro-2rectangles.lua}

\begin{enumerate} 
\item Ändere die Zahlen in Zeile 2. Was passiert?
\item Diese Farben-Repräsentation mit drei Zahlen (0-255) wird RGB (Rot-Grün-Blau) genannt. Färbe das kleinere Rechteck blau.
\item Bewege die Rechtecke, sodass sie überlappen. Welches ist oben?
\item Vertausche die Zeilen 3 und 5. Was ändert sich?
\end{enumerate}

\subsection{Ein Paar Linien}

\lstinputlisting{code/intro-lines.lua}

\begin{enumerate} 
\item Bewege das Rechteck. Passe die Linien entsprechend an.
\item Das war anstrengend. Variablen können das automatisch machen! Fahre fort.
\end{enumerate}

\subsection{Variablen}

\lstinputlisting{code/intro-variables.lua}

\begin{enumerate} 
\item Was passiert, wenn man die Zahlen für \texttt{x} und \texttt{y} vertauscht?
\item Ändere Zeile 2 zu \texttt{y = x}. Was bedeutet dies?
\item Ändere Zeile 2 zurück zu \texttt{y = 200}. Ändere Zeile 1 zu: \texttt{x = y}. Es wird eine Fehlermeldung erscheinen. Kannst Du den Code korrigieren?
\item Führe eine Variable für die Breite des Rechtecks ein.
\end{enumerate}

\section{Interaktion}

\subsection{Ein bewegtes Bild}

\lstinputlisting{code/intro-interaction.lua}

\begin{enumerate} 
\item Versuche, im Spiel den Mausknopf zu drücken. Etwas sollte geschehen.
\item Lasse das Rechteck sich rückwärts bewegen.
\item Lasse das Rechteck sich nach oben bewegen.
\item Vergrößere das Reckteck mit jedem Mausknopfdruck.
\end{enumerate}

\subsection{Wird richtig/rechts geklickt?}

\lstinputlisting{code/intro-control.lua}

\begin{enumerate} 
\item Wo muss man klicken, um das Rechteck zu bewegen?
\item Kopiere Zeile 11 und ändere sie so, dass \texttt{a} verringert wird wenn \texttt{dir == "left"}.
\item Lasse das Rechteck den Rand berühren, aber nich darüber hinaus gehen (z.B. ergänze \texttt{and a < 500} vor \texttt{then}).
\end{enumerate}

\subsection{Es soll von selbst etwas tun}

\lstinputlisting{code/intro-update.lua}

Alles innerhalb vom \texttt{love.update}-Block wird 60 mal je Sekunde ausgeführt.
\newline

\begin{enumerate} 
\item Halte das Rechteck am oberen Rand an. \texttt{if y > 200 then} ... \texttt{end} o.ä. könnte helfen.
\item Oben im Code, füge \texttt{velocity = 1} (Geschwindigkeit = 1) ein. Benutze \texttt{y = y - velocity} anstelle von \texttt{y = y - 1} um das Rechteck zu bewegen.
\item Reduziere \texttt{velocity} dauerhaft um 0.01. Das simuliert Erdanziehung.
\item Erhöhe \texttt{velocity}, wenn diese angeklickt wird (benutze \texttt{love.mousepressed()}).
\item Halte das Rechteck am unteren Rand an.
\item Zeige die Geschwindigkeit mit \texttt{love.graphics.print(velocity,10,10)}
\item Setze \texttt{velocity} gleich 0, wenn das Rechteck den oberen Rand berührt.
\item Gebe der Spielerin ein Ziel. Zeige an, wenn die Spielerin das Ziel erreicht hat. Z.B.: Ein einfaches Parken-Spiel. Zeichne eine Linie auf der Höhe 100, ändere die Farbe des Rechteckes, \texttt{if 0.5 > velocity and velocity > -0.5 and 105 > y and y > 95 then}.
\end{enumerate}

\subsection{Schleifen (Loops)}

Die \textbf{while}-Schleife (während-loop) führt das Programm in dessen Rumpf/Körper aus, so lange die Schleifen-/Laufbedingung \texttt{y < 500} wahr ist.

\lstinputlisting{code/intro-loop.lua}

\begin{enumerate}
\item Vertausche Zeilen 7 und 8. Verstehst Du den Unterschied?
\item Verschiebe Zeile 4 in Zeile 2. Der Bildschirm sollte schwarz werden. Wieso?
\item Mache die Änderungen rückgängig. Dann zeichne mehr aber kleinere Rechtecke vertikal mithilfe der while-Schleife.
\item Füge eine neue Variable \texttt{z = 0} hinzu. Füge eine neue while-Schleife hinzu. Lasse diese \texttt{z} erhöhen und Rechtecke horizontal zeichnen.
\item Bewege \texttt{z = 0} und die neue Schleife in den Körper der alten Schleife. Nun wird durch alle z-Werte hindurchgegangen, wenn \texttt{y} erhöht wird! Zeichne so ein Schachbrettmuster.
\item Bewege das Schachbrett auf Mausklick.
\end{enumerate}

\subsection{Listen}
\lstinputlisting{code/intro-array.lua}
\texttt{a} ist eine Liste (Tabelle). \texttt{a[1]} ist gleich \texttt{100}. \texttt{1} ist hier der \textit{Index} von \texttt{100} in \texttt{a}.
\newline

\begin{enumerate}
  \item Füge eine Nummer in die Liste \texttt{a} hinzu. \texttt{\#a} ist die Länge von \texttt{a}. Zeichne alle 4 Elemente.
\item \texttt{a[5] = 5 * 10} erstellt ein neues Element. Lasse mithilfe einer while-Schleife \texttt{a} gleich \texttt{\{10,20,....,200\}} sein.
\item \texttt{a[\#a+1] = v} ist das gleiche, wie \texttt{v} zur Liste hinzuzufügen. Wenn die Maus geklickt wird, füge die x-Koordinate des Klicks zur Liste hinzu.
\end{enumerate}

\iffalse
\subsection{Functions}
Functions calculate an output to a given input. They can also output nothing and just execute some code based on the input. They can also ignore their input and just be a fancy name for a block of code.
love.draw, love.update and love.mousepressed are special functions which are defined by you but executed by the game engine.\footnote{Executing a function is usually called \textit{calling a function}}

\lstinputlisting{code/intro-functions.lua}

\begin{enumerate}
\item twosquares(50,150) executes the code from lines 2-4 with x being 50 and y being 150. The values \textit{returned} in line 4 are saved in x1,y1. \newline
      Draw another pair of squares at the position returned by twosquares(x1,y1).
\item Create a new function called foursquares(x,y). This function should draw 4 squares all having some distance between them. You can use twosquares.
\item Add this function and call drawatx1y1 in love.draw.\footnote{drawatx1y1() calls the function. drawatx1y1 does not! drawatx1y1 only represents a variable holding the function.}
\begin{lstlisting}
function drawatx1y1()
  love.graphics.rectangle("fill",x1,y1,100,100)
end
\end{lstlisting}
Any code that is executed after line 8 can use x1 and y1, but often you only need a variable inside a block of code.
Sharing that variable with the rest of the code can make it more difficult to understand the code, because small changes can have widespread, hard to track, effects.

Replace line 8 by \textbf{local x1 , y1 = twosquares (50 ,150)}.

Now drawatx1y1 can not access x1 and y1 anymore. The program should crash.

How are variables transfered from love.draw to twosquares?
Can you replicate that strategy to make drawatx1y1 work again?

\item Create a function pointInRectangle(x,y,rectanglex,rectangley,width,height) which returns true when x,y is inside the rectangle drawn by love.graphics.rectangle("fill", rectanglex,rectangley,width,height) and false otherwise.
      Create a moving rectangle with love.update.
      Signify if the player has clicked the moving rectangle using pointInRectangle.
\end{enumerate}

\subsection{Object Orientation}
Objects are some functions and some variables bundled together. They can represent objects of reality like cats and trees.
The functions could then represent what a cat can do while the variables remember attributes of the cat.
This could be concievied as an object. cat[1] could be the age and "lisa"\footnote{"lisa" is a string that could be printed. Strings can be hold in variables just like numbers.} 
\begin{lstlisting}
cat = {3, "lisa", function() love.graphics.print("purr",50,50) end}
\end{lstlisting}
It would be annoying to refer to the name of the cat as cat[2] though.
Lists can actually be more than lists. They can not only associate an index with a value, but any value with any value.
For that we need new syntax.
\begin{lstlisting}
cat = {3, "lisa", function() love.graphics.print("purr",50,50) end}
\end{lstlisting}
is the same as
\begin{lstlisting}
cat = {
  3,
  "lisa",
  function() love.graphics.print("purr",50,50) end
}
\end{lstlisting}
is the same as
\begin{lstlisting}
cat = {
  1 = 3,
  2 = "lisa",
  3 = function() love.graphics.print("purr",50,50) end
}
\end{lstlisting}
but you can do
\begin{lstlisting}
cat = {
  age = 3,
  name = "lisa",
  askforfood = function() love.graphics.print("purr",50,50) end
}
\end{lstlisting}
now cat["age"] is 3 and cat["name"] is lisa.
Beware that cat["age"] is not the same cat[age]. What cat[age] is, depends on what value the variable age has.

One can create a function to make the creation of complicated objects easier.

\begin{lstlisting}
function newCat(age, name)
return {
    age = age,
    name = name,
    askforfood = function() love.graphics.print("purr",50,50) end
  }
end
\end{lstlisting}
Here's how to give the functions  access the objects variables.\footnote{\textbf{a .. b} appends the string b to a. Instead of cat.askforfood(cat) you can write cat:askforfood(). Its the same.}
\begin{lstlisting}
function newCat(age, name)
  return {
    age = age,
    name = name,
    askforfood = function(thecat)
      love.graphics.print("purr I'm " .. thecat.name ,50,50)
    end
  }
end

cat = newCat(4,"tom")

function love.draw()
  cat.askforfood(cat)
end
\end{lstlisting}

\lstinputlisting{code/intro-objects.lua}

\fi

\end{document}
